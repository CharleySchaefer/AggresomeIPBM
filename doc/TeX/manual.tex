
\documentclass[a4paper, 11 pt]{article}

\usepackage[pdftex]{graphicx}

\usepackage[english]{babel}
\usepackage[latin1]{inputenc}
\usepackage{a4wide}

\usepackage{epstopdf}			% Use eps figures
\usepackage{color}
\usepackage{amsmath} % split environment
\usepackage{amssymb} % split environment
\usepackage{amsfonts} % split environment
\usepackage{mathtools} % multlined environment
\usepackage{ctable}
\usepackage{tabularx}

\usepackage{amsthm}
\usepackage{ctable}
\usepackage{wrapfig}
\usepackage[small,bf]{caption}
\usepackage[T1]{fontenc}
\usepackage{ae,aecompl}
\usepackage{url}
\usepackage{float}
\usepackage{ifthen}

\usepackage{varioref}
\usepackage{hyperref}
\usepackage{cleveref}

\title{Manual}
\author{Charley Schaefer}
\bibliography{references}
\begin{document}
 \maketitle
 \date{}
  \section{Diffusion 1D}
  Protein can hop a distance $\Delta x$ to the left or right; both with rate $\nu$.
  For a time $\Delta t<1/2\nu$, the probability of moving either left or right is $p=2\nu\Delta t$. The mean displacement is $(1-p)\times 0 + (p/2)(\Delta x) +(p/2)(-\Delta x)=0$, with $1-p$ the probability that the displacement is $0$, $p/2$ the probability of either moving left or right.
The mean square displacement is $(1-p)\times 0^2 + (p/2)(\Delta x)^2 +(p/2)(-\Delta x)^2=p(\Delta x)^2 = 2\nu (\Delta x)^2 \Delta t$.
  If we consider $N$ subsequent time steps, $\Delta t_i$, with $i=1,2,\dots N$, the mean square displacement is
\begin{equation}
  \mathrm{MSD}(t) = \sum_{i} 2\nu (\Delta x)^2 \Delta t_i
= 2\nu (\Delta x)^2 t,
\end{equation}
with $t\equiv \sum_{i=1}^{N} \Delta t_i$. We conclude $\nu (\Delta x)^2 = D$ is the diffusivity.


\end{document}
